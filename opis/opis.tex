\documentclass[12pt,a4paper]{amsart}
% ukazi za delo s slovenscino -- izberi kodiranje, ki ti ustreza
\usepackage[slovene]{babel}
\usepackage[utf8]{inputenc}
%\usepackage[T1]{fontenc}
\usepackage{amsmath,amssymb,amsfonts}
\usepackage{url}
%\usepackage[normalem]{ulem}
\usepackage[dvipsnames,usenames]{color}
\usepackage{caption}
\usepackage{lipsum}
\usepackage{tikz}

\usetikzlibrary{graphs}
\usetikzlibrary{graphs.standard}


% ne spreminjaj podatkov, ki vplivajo na obliko strani
\textwidth 15cm
\textheight 24cm
\oddsidemargin.5cm
\evensidemargin.5cm
\topmargin-5mm
\addtolength{\footskip}{10pt}
\pagestyle{plain}
\overfullrule=15pt % oznaci predlogo vrstico


% ukazi za matematicna okolja
\theoremstyle{definition} % tekst napisan pokoncno
\newtheorem{definicija}{Definition}[section]
\newtheorem{primer}[definicija]{Example}
\newtheorem{opomba}[definicija]{Remark}

\renewcommand\endprimer{\hfill$\diamondsuit$}

\theoremstyle{plain} % tekst napisan posevno
\newtheorem{lema}[definicija]{Lemma}
\newtheorem{izrek}[definicija]{Theorem}
\newtheorem{trditev}[definicija]{Statement}
\newtheorem{posledica}[definicija]{Corollary}
\newtheorem{conjecture}[definicija]{Conjecture}


% za stevilske mnozice uporabi naslednje simbole
\newcommand{\R}{\mathbb R}
\newcommand{\N}{\mathbb N}
\newcommand{\Z}{\mathbb Z}
\newcommand{\C}{\mathbb C}
\newcommand{\Q}{\mathbb Q}

% ukaz za slovarsko geslo
\newlength{\odstavek}
\setlength{\odstavek}{\parindent}
\newcommand{\geslo}[2]{\noindent\textbf{#1}\hspace*{3mm}\hangindent=\parindent\hangafter=1 #2}

% naslednje ukaze ustrezno popravi
\newcommand{\program}{Finančna matematika} % ime studijskega programa: Matematika/Finančna matematika
\newcommand{\imeavtorja}{Žiga Mazej} % ime avtorja
\newcommand{\imementorja}{doc. dr. Janoš Vidali} % akademski naziv in ime mentorja
\newcommand{\imesomentorja}{prof. dr. Riste Škrekovski}
\newcommand{\naslovdela}{Sigma total irregularity of bipartite graphs}
\newcommand{\letnica}{2023} %letnica diplome

\begin{document}

\thispagestyle{empty}
\noindent{\large
UNIVERZA V LJUBLJANI\\[1mm]
FAKULTETA ZA MATEMATIKO IN FIZIKO\\[5mm]
\program\ }
\vfill

\begin{center}{\large
\imeavtorja\\[2mm]
{\bf \naslovdela}\\[10mm]
Skupinski projekt\\[2mm]
Kratek opis\\[1cm]
Advisers: \imementorja, \\ \imesomentorja\\[2mm]}
\end{center}
\vfill

\noindent{\large
Domžale, \letnica}
\pagebreak

\section{Navodilo naloge}

\noindent
Želimo najti dvodelne grafe reda $n$ z največjo možno sigma totalno iregularnostjo. Da bi dosegli to, naredimo naslednje:

\begin{enumerate}
    \item Prvič, za majhne vrednosti $n$ poiščimo optimalne grafe z uporabo sistematičnega iskanja.
    \item Drugič, poskusimo posplošiti svoje ugotovitve za večje $n$ in jih podrobno preizkusimo.
    \item Navedimo natančno izjavo o tem, kateri graf/i je/so optimalen/ni, in preizkusimo svojo domnevo s spreminjanjem kandidatov,
     tj., vedno bi morali dobiti graf z manjšo sigma totalno iregularnostjo. Tu lahko uporabimo nekaj metahevristike.

\end{enumerate}




\section{Opis problema}

Skušam najti dvodelni graf z maksimalno sigma totalno iregularnostjo, glede na red n. Problema se bom lotil s simatičnim iskanjem za majhne n,
nadaljeval za večje n, na koncu pa bom poskušal najti čim bolj optimalne grafe za velike n z metahevristiko.


\section{Potek Dela}
Najprej bom iskal sigma totalno iregularnost, ki jo definira enačba
\begin{align}
    \sigma_t(G) = \sum_{\{u,v\} \subseteq V(G)} (d_G(u) - d_G(v))^2
\end{align}
za dvodelne grafe nizkega reda. Red 2 je trivialen. Nasledjnih nekaj redov se bo dalo preveriti ročno oziroma sistematično.
Izmed teh bom izbral tiste, katerih vrednosti sigma totalne iregularnosti bodo največje in poskušal iskati nek vzorec
ter algoritem, ki bi deloval na večjih redih n. Ustreznost in delovanje tega algoritma bom preveril v Sage-u.
To bom storil tako da bom z majhnimi odstopanji od optimalnih grafov, najdenih s pomočjo že omenjenega
algoritma, želel vedno dobiti manjšo vrednost sigma totalne iregularnosti. Tu si bom pomagal tudi z metahevristiko.

\end{document}